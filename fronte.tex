%\begin{equation}
%\end{equation}
%\clearpage


\newpage
\pagestyle{empty}
%\begin{titlepage}
\begin{figure}[h]
\centering
\includegraphics[width=7cm]{logo_mines_paristech}
%\end{center}
\end{figure}
\vspace{0.25cm}
\begin{center}
\fontsize{10pt}{10pt}\selectfont
    \textbf{Universit\`{a} degli Studi di Pavia}\\
    \vspace{0.3cm}
    \fontsize{8pt}{8pt}\selectfont
    \textbf{Facolt\`{a} di Ingegneria}\\
\end{center}



\vspace{1cm}

\begin{center}
    \fontsize{20pt}{20pt}\selectfont
    \textbf{ANALISI ISOGEOMETRICA}\\
    \fontsize{16pt}{16pt}
    \textbf{Vantaggi derivanti dall'utilizzo}\\
    \textbf{di funzioni di forma a continuit\`{a} elevata}\\
    \textbf{per problemi monodimensionali}\\
    
    \vspace{0.75cm}

    \fontsize{14pt}{14pt}\selectfont
    \vspace{0.5cm}
    \textbf{Luca Marioni}
    \vspace{0.5cm}

    \fontsize{12pt}{12pt}\selectfont
      Tesi per il conseguimento \\della laurea in:
    \vspace{0.25cm} 
    \textbf{Ingegneria civile e ambietale}\\ 
    presso l'Universit\`{a} degli Studi di Pavia\\ 
 \fontsize{12pt}{12pt}\selectfont
      \textbf{relatore:}\\ 
      Prof. A. Reali - Universit\`{a} degli Studi di Pavia\\
 
      \fontsize{12pt}{12pt}\selectfont
      \textbf{correlatore:}\\ 
      Prof.ssa L.D. Marini - Universit\`{a}  degli Studi di Pavia\\



    \end{center}
%\end{titlepage}
\cleardoublepage

%%%%%%%%%%%%%%%%%%%%%%%%%%%%%%%%%%%%%%%%%%%%%%%%%%%%

%\thispagestyle{empty}
%\begin{flushright}
%\null\vspace{\stretch{1}}
%{\emph{Il linguaggio della matematica si rivela\\ irragionevolmente efficace nelle scienze naturali;\\ un dono meraviglioso che non comprendiamo n\'e meritiamo}\\
%\footnotesize{E. Wigner}}
%\end{flushright}
%\cleardoublepage


\newpage
\pagestyle{empty}


\chapter*{Abstract}


\chapter*{Sommario}



