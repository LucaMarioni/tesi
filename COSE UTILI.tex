 \begin{figure} [t!]
 \centering
 \subfigure[Coefficienti pariziali di sicurezza sulle azioni]
   {\includegraphics[width=.5\textwidth]{A}}
 \hspace{5mm}
 \subfigure[Coefficienti pariziali di sicurezza sui parametri dei materiali]
   {\includegraphics[width=.5\textwidth]{M}}
    \vspace{5mm}
    \centering
    \subfigure[Coefficienti pariziali di sicurezza sulle resistenze globali]
      {\includegraphics[width=.6\textwidth]{R}}
 \caption{Coefficienti parziali di sicurezza} \label{ntc}
 \end{figure}
 % % % % % % % % % % % % % % % % % % % % % % % % % % % % % % % % % % % % % % % % % % % % % % % % % % % % % % %
 utf8x------>LETTERE ACCENTATE
 % % % % % % % % % % % % % % % % % % % % % % % % % % % % % % % % % % % % % % % % % % % % % %
 % % % % % % % % % % % % % % % % % % % % % % % % % % % % % % % % % % % % % % % % %
 \begin{equation}
 \begin{split}
 &N_d=550\cdot 1,3+180\cdot 1,3+350 \cdot 1,5=1474\,kN \\
 &M_d=70\cdot 1,3+ 25 \cdot 1,5=128,5\,kNm \\
 &H_d=45 \cdot 1,3+ 15 \cdot 1,5= 81 \,kN
 \end{split}
 \end{equation}
 % % % % % % % % % % % % % % % % % % % % % % % % % % % % % % % % % % % % % % % % % % % % % % % % % % % % % % % % % %
\begin{tabular}{|l|c|c|}
\hline
ESTREMO & Taglio [kN] & Momento [kNm]   \\
\hline
Sinistra       & 161     & -2 \\
\hline
Destra         & -123     & 11  \\
\hline
\end{tabular}
% % % % % % % % % % % % % % % % % % % % % % % % % % % % % % % % % % % % % % % % % % % % % % % % % % % % % % % % % % % %
\[ |x| = \left\{ \begin{array}{ll}
         x & \mbox{if $x \geq 0$};\\
        -x & \mbox{if $x < 0$}.\end{array} \right. \] 
% % % % % % % % % % % % % % % % % % % % % % % % % % % % % % % % % % % % % % % % % % % % % % % % % % % % % % % % % %
pdflatex + F11 + pdf +pdf: BIBLIOGRAFIA
% % % % % % % % % % % % % % % % % % % % % % % % % % % % % % % % % % % % % % % % % % % % % % % % % % % % % % % % % % % % % %
\listoftables
% % % % % % % % % % % % % % % % % % % % % % % % % % % % % % % % % % % % % % % % % % % % % % % % % % % % % % % % % % %
\begin{table} [h!]
  \caption{Calcolo $ \sigma_s $} \label{sigmas}
\centering
  \begin{tabular}{|l|c|c|c|c|c|c|}   
  \hline
    & & & \multicolumn{2}{|c|}{C. Frequente} & \multicolumn{2}{|c|}{C. Quasi Permanente}   \\
  \hline
Sez.&$ M_{cr}^{'}[kNm] $&$ M_{cr}^{''}[kNm] $&$M[kNm]$&$ \sigma_s[kPa] $&$M[kNm]$&$ \sigma_s[kPa] $\\
\hline
AA&35&33&18&29&17&27\\
BB&33&31&15,5&24&12&19\\
CC&38&36&44,5&178&42&168\\
DD&33&31&26&42&22&35\\
EE&36&34&31&45&29&42\\
\hline

  \end{tabular} 
  \end{table}
  % % % % % % % % % % % % % % % % % % % % % % % % % % % % % % % % % % % % % % % % % % % % % % % % % % % % % % % % % %
  \eqref \vref \ref \cite \citet \citep
  % % % % % % % % % % % % % % % % % % % % % % % % % % % % % % % % % % % % % % % % % % % % % % % % % %
  \begin{equation}
  l_0=max
  \left\{
  \begin{aligned}
  \label{nomechevuoitu}
  &(\sum_{i=1}^{n} x_i^2)a + (\sum_{i=1}^{n} x_i)b = \sum_{i=1}^{n} x_i y_i \\
  &(\sum_{i=1}^{n} x_i)a + nb = \sum_{i=1}^{n} y_i
  \end{aligned}
  \right.
  \end{equation} 